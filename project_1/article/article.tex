\documentclass[
    a4paper, aps, twocolumn, floatfix, superscriptaddress, nofootinbib]{revtex4-1}

% It's nice to be able to write your own name
\usepackage[T1]{fontenc}
% Automatic clickable links
\usepackage{hyperref}
% SI-units
\usepackage{siunitx}
% Enhanced math formatting
\usepackage{amsmath}
% Extended math symbols
\usepackage{amssymb}
% Include proof environment
\usepackage{amsthm}
% Import physics package to include bra-ket
\usepackage{physics}
\usepackage{enumerate}
% Import the tensor package for tensors
\usepackage{tensor}
% Include dod and dpd fracs
\usepackage{commath}
% Include font for the identity operator
\usepackage{dsfont}
% Include package for drawing Feynman diagrams
\usepackage{tikz-feynman}
% Include package for typesetting contractions
\usepackage{simpler-wick}
% Include the Feynman-slash
\usepackage{slashed}
% Include Tikz
\usepackage{tikz}


% Macro for latin-letter vectors
\newcommand{\vf}{\mathbf}
% Macro for greek-letter vectors
\newcommand{\vfg}{\boldsymbol}

% Fast macro for real-numbers R
\newcommand{\R}{\mathbb{R}}
% Fast macro for complex-numbers C
\newcommand{\C}{\mathbb{C}}
% Fast macro for polynomial room P
\renewcommand{\P}{\mathbb{P}}
% New command for the identity operator
\newcommand{\1}{\mathds{1}}
% New command for the Lagrangian density
\newcommand{\cL}{\mathcal{L}}
% New command for the Hamiltonian density
\newcommand{\cH}{\mathcal{H}}
% Fast macro for partial differential tensors
\newcommand{\tpl}[1]{\tensor{\partial}{_#1}} % Lower
\newcommand{\tpu}[1]{\tensor{\partial}{^#1}} % Upper
% Fast macro for tensors
\newcommand{\te}[1]{\tensor{#1}}
% Fast macro for commutation and anti-commutation relations
\newcommand{\com}[2]{\left[#1, #2\right]}
\newcommand{\acom}[2]{\left\{#1, #2\right\}}

% Macros for writing auto-sized paranthesis, brackets and braces
\newcommand{\para}[1]{\left(#1\right)}
\newcommand{\brak}[1]{\left[#1\right]}
\newcommand{\brac}[1]{\left\{#1\right\}}

% Macro for creating orbital bra-ket's, i.e., bra-ket with paranthesis edges
\newcommand{\obra}[1]{( #1 \lvert}
\newcommand{\oket}[1]{\rvert #1 )}
\newcommand{\obraket}[2]{( #1 \lvert #2 )}

% Macro for expectation value
\newcommand{\expv}[1]{\langle #1 \rangle}

\newcommand{\half}{\frac{1}{2}}




\begin{document}

\title{Variational Monte Carlo on bosonic systems}
\author{Winther-Larsen, Sebastian Gregorius}
\affiliation{University of Oslo}
\author{Schøyen, Øyvind Sigmundson}
\affiliation{University of Oslo}
\date{\today}

\begin{abstract}
    Something very abstract and clever should go here.
    \begin{figure}[h!]
    \centering
    \includegraphics[width=0.7\textwidth]{loopholes.jpg}
\end{figure}
\end{abstract}

\maketitle

%\begin{widetext}

%\end{widetext}

\section{Introduction}
    We will in this project study the Variational Monte Carlo (VMC) method, and
    use it to evaluate the ground state energy of a trapped, hard sphere Bose
    gas.

\section{Theory}
    To model the trapped bosonic gas particles we use the potential
    \begin{align}
        V_{\text{ext}}(\vf{r})
        &=
        \begin{cases}
            \half m\omega^2r^2 & (\text{S}), \\
            \half m \bigl[
                \omega^2(x^2 + y^2) + \omega_z^2z^2
            \bigr] & (\text{E}),
        \end{cases}
    \end{align}
    where we can choose between a spherical (S) or an elliptical (E) harmonic
    trap. The two-body Hamiltonian of the system is given by
    \begin{align}
        H = \sum_{i = 1}^{N}h(\vf{r}_i) + \sum_{i < j}^{N}w(\vf{r}_i, \vf{r}_j),
    \end{align}
    where the single particle one body operator, $h$, is given by
    \begin{align}
        h(\vf{r}_i) = -\frac{\hbar^2}{2m}\nabla_i^2
        + V_{\text{ext}}(\vf{r}_i),
    \end{align}
    (we assume equal mass)
    and the two-body interaction operator, $w$, is
    \begin{align}
        w(\vf{r}_i, \vf{r}_j)
        = \begin{cases}
            \infty & |\vf{r}_i - \vf{r}_j| \leq a, \\
            0 & |\vf{r}_i - \vf{r}_j| > a,
        \end{cases}
    \end{align}
    where $a$ is the hard sphere of the particle. The trial wavefunction,
    $\ket{\Psi_T}$, we will
    be looking at is given by
    \begin{align}
        \Psi_T(\vf{r})
        &= \Phi_T(\vf{r})
        \prod_{j < k}^N f(a, \vf{r}_j, \vf{r}_k) \\
        &= \Biggl(
            \prod_{i = 1}^N g(\alpha, \beta, \vf{r}_i)
        \Biggr)
        \prod_{j < k}^N f(a, \vf{r}_j, \vf{r}_k),
        \label{eq:initial_trial_wavefunction}
    \end{align}
    where $\alpha$ and $\beta$ are variational parameters and
    \begin{align}
        \vf{r} = (\vf{r}_1, \vf{r}_2, \dots, \vf{r}_N, \alpha, \beta).
    \end{align}
    Here $g$ are the single particle wavefunctions given by
    \begin{align}
        g(\alpha, \beta, \vf{r}_i)
        = \exp\bigl[
            -\alpha(x_i^2 + y_i^2 + \beta z_i^2)
        \bigr] \equiv \phi(\vf{r}_i),
    \end{align}
    and $\ket{\Phi_T}$ the \textit{Slater permanent} consisting of the $N$ first
    single particle wavefunctions, and the correlation wavefunction, $f$, given
    by
    \begin{align}
        f(a, \vf{r}_j, \vf{r}_k)
        &=
        \begin{cases}
            0 & |\vf{r}_j - \vf{r}_k| \leq a, \\
            \Bigl(
                1 - \frac{a}{|\vf{r}_j - \vf{r}_k|}
            \Bigr) & |\vf{r}_j - \vf{r}_k| > a.
        \end{cases}
    \end{align}
    We will for brevity use the notation $\phi(\vf{r}_i) = \phi_i$ and $r_{jk} =
    |\vf{r}_j - \vf{r}_k|$.

    \subsection{Local energy}
        As the many-body wavefunction creates a very large configuration space,
        where much of the wavefunction is small, we use the Metropolis algorithm
        in order to move towards regions in configuration space with
        ``sensible'' values. We define the \textit{local energy}, $E_L{\vf{r}}$,
        by
        \begin{align}
            E_L(\vf{r})
            &= \frac{H\Psi_T(\vf{r})}{\Psi_T(\vf{r})}.
        \end{align}
        If $\ket{\Psi_T}$ is an exact eigenfunction of the Hamiltonian, $E_L$
        will be constant. The closer $\ket{\Psi_T}$ is to the exact wave
        function, the less variation in $E_L$ as a function of $\vf{r}$ we get.
        One of the most computationally intensive parts of the VMC algorithm
        will be to compute $E_L$. We therefore find an analytical expression for
        $E_L$ in terms of the trial wavefunction.

    \subsection{The drift force}
        A disadvantage in the use of the brute-force Metropolis algorithm is
        that we might be spending much computational resources in an
        uninteresting part of configuration space. To make smarter moves we will
        use the Metropolis-Hastings algorithm (which will be discussed in due
        time). This algorithm is dependent on the drift force of the system.
        \begin{align}
            \vf{F}(\vf{r})
            &=
            \sum_{k = 1}^N
            \vf{F}_k(\vf{r})
            =
            \sum_{k = 1}^N
            \frac{2\nabla_k\Psi_T(\vf{r})}{\Psi_T(\vf{r})}.
        \end{align}
        Using this expression we are able to move towards parts of
        configuration space where the gradient increases or decreases yielding a
        better choice of movements. We will mainly be interested in the drift
        force of a single particle $k$.

\section{Non-interacting harmonic oscillators}
    We start by looking at a simple system of non-interacting harmonic
    oscillators. That is, where $a = 0$ and $\beta = 1$. We thus get the trial
    wavefunction
    \begin{align}
        \Psi_T(\vf{r})
        = \Phi_T(\vf{r})
        = \prod_{i = 1}^N \exp\bigl[
            -\alpha |\vf{r}_i|^2
        \bigr],
    \end{align}
    where $|\vf{r}_i| = r_i$. As $a = 0$ the interaction term,
    $w(\vf{r}_i, \vf{r}_j)$, vanishes and the Hamiltonian is given by
    (in the spherical case)
    \begin{align}
        H &= \sum_{i = 1}^N h(\vf{r}_i)
        = \sum_{i = 1}^N \Biggl(
            -\frac{\hbar^2}{2m}\nabla_i^2
            + \half m \omega^2 r_i^2
        \Biggr).
    \end{align}
    To find the drift force and the local energy we have to compute the gradient
    and the Laplacian of the trial wavefunction. The gradient is given by
    \begin{align}
        \nabla_k\Psi_T(\vf{r})
        &= -2\alpha \vf{r}_k\Psi_T(\vf{r}),
    \end{align}
    whereas the Laplacian yields
    \begin{align}
        \nabla^2_k\Psi_T(\vf{r})
        &= \big(-2d\alpha + 4\alpha^2 r_k^2\bigr)\Psi_T(\vf{r}),
    \end{align}
    where $d$ is the dimensionality of the problem determined by
    $\vf{r}_k \in \mathbb{R}^d$. We can thus use the gradient to find an
    expression for the drift force for particle $k$.
    \begin{align}
        \vf{F}_k(\vf{r})
        &= -2\alpha\vf{r}_k.
    \end{align}
    Using the Laplcian we can compute the kinteic term in the expression for the
    local energy. We get
    \begin{align}
        E_L(\vf{r})
        &=
        \sum_{i = 1}^N
        \Biggl(
            -\frac{\hbar^2}{2m}
            \bigl[
                -2d\alpha + 4\alpha^2 r_i^2
            \bigr]
            + \half m\omega^2 r_i^2
        \Biggr).
    \end{align}
    In natural units, with $\hbar = c = m = 1$, this reduces to
    \begin{align}
        E_L(\vf{r})
        &=
        \alpha dN
        + \biggl(
            \half\omega^2
            - 2\alpha^2
        \biggr)
        \sum_{i = 1}^N r_i^2.
    \end{align}
    It is worth noting that for $\alpha = \pm\half\omega$ we will find a stable
    value which turns out to be the exact energy minimum. This happens as the
    entire sum over all the random walkers disappears.


\section{Interacting hard sphere bosons}
    Moving to the full system allowing $\beta$ to vary and setting $a
    \neq 0$ we can write the trial wavefunction as
    \begin{align}
        \Psi_T(\vf{r})
        &=
        \Phi_T(\vf{r})
        J(\vf{r}),
    \end{align}
    where $\ket{\Phi_T}$ is the same Slater permanent as in
    \autoref{eq:initial_trial_wavefunction} and $J(\vf{r})$ is the
    \textit{Jastrow factor} given by
    \begin{align}
        J(\vf{r})
        &=
        \exp\Biggl(
            \sum_{j < l}^N u(r_{jl})
        \Biggr),
    \end{align}
    where $r_{jk} = |\vf{r}_j - \vf{r}_k|$ and
    \begin{align}
        u(r_{jk}) = \ln\bigl[f(a, \vf{r}_j, \vf{r}_k)\bigr].
    \end{align}
    To further shorten the notation we will use $u_{jk} = u(r_{jk})$. Computing
    the gradient of the wavefunction we get
    \begin{align}
        \nabla_k\Psi_T(\vf{r})
        &=
        \Bigl[
            \nabla_k
            \Phi_T(\vf{r})
        \Bigr]
        J(\vf{r})
        + \Phi_T(\vf{r})
        \nabla_k J(\vf{r}).
    \end{align}
    The gradient of the Slater permament for particle $k$ is given by
    \begin{align}
        \nabla_k
        \Phi_T(\vf{r})
        &=
        \nabla_k\phi_k
        \prod_{i \neq k}^N\phi_i
        = \frac{\nabla_k\phi_k}{\phi_k}
        \Phi_T(\vf{r}).
    \end{align}
    The gradient of the Jastrow factor is given by
    \begin{align}
        \nabla_k J(\vf{r})
        &=
        J(\vf{r})
        \nabla_k\sum_{m < n}^N u_{mn} \\
        &= J(\vf{r})
        \Biggl(
            \sum_{m = 1}^{k - 1}\nabla_k u_{mk}
            \sum_{n = k + 1}^N\nabla_k u_{kn}
        \Biggr)
        \\
        &=
        J(\vf{r})
        \sum_{m \neq k}^N\nabla_k u_{km},
    \end{align}
    where the gradient of the interaction term splits the antisymmetric
    sum into two parts. As $r_{ij} = r_{ji}$ we can combine these sums
    into a single sum. This in total yields the gradient
    \begin{align}
        \nabla_k\Psi_T(\vf{r})
        &=
        \Biggl(
            \frac{\nabla_k\phi_k}{\phi_k}
            + \sum_{m \neq k}^N
            \nabla_k u_{km}
        \Biggr)
        \Psi_T(\vf{r}).
        \label{eq:gradient_full_wavefunction}
    \end{align}

    The Laplcian of the trial wavefunction is found by finding the divergence of
    \autoref{eq:gradient_full_wavefunction}.
    \begin{align}
        \nabla_k^2\Psi_T(\vf{r})
        &=
        \Biggl(
            \nabla_k\Biggl[
                \frac{\nabla_k\phi_k}{\phi_k}
            \Biggr]
            +
            \sum_{m \neq k}^N \nabla_k^2 u_{km}
        \Biggr)\Psi_T(\vf{r})
        \\
        &\qquad
        +
        \Biggl(
            \frac{\nabla_k\phi_k}{\phi_k}
            + \sum_{m \neq k}^N
            \nabla_k u_{km}
        \Biggr)^2
        \Psi_T(\vf{r}),
    \end{align}
    where the squared term came from taking the gradient of the trial
    wavefunction.  To further simplify we divide by the trial
    wavefunction. This yields
    \begin{align}
        \frac{\nabla_k^2\Psi_T(\vf{r})}{\Psi_T(\vf{r})}
        &=
        \frac{\nabla_k^2\phi_k}{\phi_k}
        + 2\frac{\nabla_k\phi_k}{\phi_k}
        \sum_{m \neq k}\nabla_k u_{km}
        \nonumber \\
        &\qquad
        + \sum_{m\neq k}^N\nabla_k^2 u_{km}
        + \Biggl(
            \sum_{m \neq k}^N\nabla_k u_{km}
        \Biggr)^2.
    \end{align}
    To go from here we have to find the gradient and the Laplacian of
    the single particle functions, $\phi_k$, and the interaction
    functions $u_{km}$. For the single particle functions we use
    Cartesian coordinates when finding the derivatives whereas we for
    the interaction functions will use spherical coordinates and do a
    variable substitution. Beginning with the gradient of the single
    particle functions we get
    \begin{align}
        \nabla_k\phi_k
        &=
        \nabla_k\exp\bigl[
            -\alpha(x_k^2 + y_k^2 + \beta z_k^2)
        \bigr] \\
        &=
        -2\alpha
        (x_k\vf{e}_i + y_k\vf{e}_j + \beta z_k\vf{e}_k)
        \phi_k.
    \end{align}
    Note that the subscripts on the unit vectors $\vf{e}_i$ are
    \textit{not} the same as the subscripts used for its components. The
    Laplacian yields
    \begin{align}
        \nabla_k^2\phi_k
        &=
        \Bigl[
            -2\alpha
            \big(
                d - 1 + \beta
            \bigr)
            \nonumber \\
            &\qquad
            + 4\alpha^2
            \bigl(
                x_k^2 + y_k^2 + \beta^2z_k^2
            \bigr)
        \Bigr]
        \phi_k,
    \end{align}
    with $d$ as the dimensionality of the problem.
    In order to derive the interaction functions we have to do a
    variable substitution using $r_{km} = |\vf{r}_k - \vf{r}_m|$. We can
    then rewrite the $\nabla_k$-operator as
    \begin{align}
        \nabla_k
        &=
        \nabla_k
        \dpd[]{r_{km}}{r_{km}}
        =
        \nabla_k r_{km} \dpd[]{}{r_{km}}
        \\
        &=
        \frac{\vf{r}_k - \vf{r}_m}{r_{km}}\dpd[]{}{r_{km}}.
    \end{align}
    Applying this version of the $\nabla_k$-operator to $u_{km}$ yields
    \begin{align}
        \nabla_k u_{km}
        &=
        \frac{\vf{r}_k - \vf{r}_m}{r_{km}}
        \dpd[]{u_{km}}{r_{km}}.
    \end{align}

    \begin{widetext}
        For the Laplacian we switch a little back and forth between the
        two ways of representing the $\nabla_k$-operator. We thus get
        \begin{align}
            \nabla_k^2 u_{km}
            &=
            \frac{\nabla_k \vf{r}_k}{r_{km}}\dpd[]{u_{km}}{r_{km}}
            + \Biggl[
                \nabla_k \frac{1}{r_{km}}
            \Biggr]
            (\vf{r}_k - \vf{r}_m)\dpd[]{u_{km}}{r_{km}}
            + \frac{\vf{r}_k - \vf{r}_m}{r_{km}}
            \nabla_k \dpd[]{u_{km}}{r_{km}} \\
            &= \frac{d}{r_{km}}\dpd[]{u_{km}}{r_{km}}
            - \frac{(\vf{r}_k - \vf{r}_m)^2}{r_{km}^3}
            \dpd[]{u_{km}}{r_{km}}
            + \frac{(\vf{r}_k - \vf{r}_m)^2}{r_{km}^2}
            \dpd[2]{u_{km}}{r_{km}} \\
            &=
            \frac{d - 1}{r_{km}}\dpd[]{u_{km}}{r_{km}}
            + \dpd[2]{u_{km}}{r_{km}},
        \end{align}
        where $d$ is again the dimensionality of the problem. In total
        we can state an intermediate version of the Laplacian occuring
        in the local energy as
        \begin{align}
            \frac{\nabla_k^2\Psi_T(\vf{r})}{\Psi_T(\vf{r})}
            &=
            \frac{\nabla_k^2\phi_k}{\phi_k}
            + 2\frac{\nabla_k\phi_k}{\phi_k}
            \sum_{m \neq k}^N
            \frac{\vf{r}_k - \vf{r}_m}{r_{km}}
            \dpd[]{u_{km}}{r_{km}}
            + \sum_{m\neq k}^N
            \Biggl(
                \frac{d - 1}{r_{km}}\dpd[]{u_{km}}{r_{km}}
                + \dpd[2]{u_{km}}{r_{km}}
            \Biggr)
            \nonumber \\
            &\qquad
            +
            \sum_{m, n \neq k}^N
            \frac{\vf{r}_k - \vf{r}_m}{r_{km}}
            \frac{\vf{r}_k - \vf{r}_n}{r_{kn}}
            \dpd[]{u_{km}}{r_{km}}
            \dpd[]{u_{kn}}{r_{kn}}.
        \end{align}
    \end{widetext}
    Moving on to the derivatives of the interaction terms, $u_{km}$, to
    get an explicit expression for the Laplacian.
    \begin{align}
        \dpd[]{u_{km}}{r_{km}}
        &=
        \frac{a}{r_{km}(r_{km} - a)},
        \\
        \dpd[2]{u_{km}}{r_{km}}
        &= \frac{a^2 - 2ar_{km}}{r_{km}^2(r_{km} - a)^2}.
    \end{align}
    The local energy and the drift force can now be found by combining these
    expressions. For brevity, we will not write out the explicit expressions as
    these will be called by separated functions in our programs.


\section{Algorithms}

In the project we rely on a Monte Carlo approach of random sampling to obtain numerical results.
Ironic, is it not? To rely randomness to solve problems that must be naturally deterministic?
We simulate random walks over a volume in order to find optimal parameters in our trial wavefunctions.
The most common of such methods, which we make use of herein, is the Metropolis-Hastings algorithm.

\subsection{Metropolis-Hastings Algorithm}
The Metropolis-Hastings algorithm can in our particular situation be condensed down to the following steps,
\begin{enumerate}
    \item The system is initialised by a certain number $N$ of randomly generated positions, or particles. This allows us to evaluate the
     wavefunction at these points and compute the local energy $E_L$.
    \item The initial configuration is changed by setting a new position for one of these particles. The particle is picked at random.
    \item A ratio between new wavefunction density and the previous (initial) density is computed and compared to a random number.
    This acceptance probability decides if the particle move is rejected or accepted.
    The particle is only allowed to move a predetermined step length.
    \item If the particle movement is accepted and the local energy $E_L$ is computed for the new system.
    \item Repeat steps until convergence and an optimum is reached.
\end{enumerate}

The algorithm described above can be applied in an "exhaustive" search of the parameter space in order to find the optimal parameters.
Whether a proposed move is accepted or not is determined by a transition probability and the acceptance probability.
The strength of the algorithm is that the transition algorithm need not be known.

\subsubsection{Importance Sampling}
A problem with the naïve Metropolis-Hastings sampling approach is that the sampling of the position space is done with
no regard for where we are likely to find a particle. This problem can be remedied through a \emph{importance sampling}.
It is reasonable to assume that the particles we erratically scatter in space are prone to movement towards the peaks
of the probability density as dictated by the wave function. Consider therefore the Fokker-Planck equation,
\begin{equation}
    \frac{\partial \psi}{\partial t} = D \frac{\partial }{\partial x}\left(\frac{\partial }{\partial x} - F \right) \psi,
    \label{eq:fokker_planck}
\end{equation}
which described the evolution in time of a probability density function, here exchanged for the wavefunction $\psi$.
Originaly an equation that models diffusion, we have a diffusion term $D$ and a drift force,
\begin{equation}
    F = \frac{2}{\psi} \frac{\partial \psi}{\partial x}.
\end{equation}

The one-particle model corresponding to the Fokker-Planck equation is the Langevin equation,
\begin{equation}
    \frac{\partial x}{\partial t} = D F(x) + \eta,
    \label{eq:langevin}
\end{equation}
where $\eta$ is a uniformly distributed stochastic variable. Solving Langevin's equation by Eurler's method gives a recursive relation for
the subsequent new positions of a particle,
\begin{equation}
    x_{t+\Delta t} = x_t + D F(x) \Delta t +\xi \sqrt{\Delta t},
\end{equation}
given a time step $\Delta t$\footnote{Bear in mind that \autoref{eq:langevin} is only valid as $\Delta t \to 0$, a property stemming from the use of Euler's method.} and a normally distributed stochastic variable $\xi$.

Now we need to change the acceptance probability of the metropolis algorithm to something that takes
the new sampling method into account,
\begin{equation}
    q(x_{\text{new}}, x) = \frac{G(x, x_{\text{new}}, \Delta t)\abs{\psi_T(x_{\text{new}})}^2}{G(x_{\text{new}}, x, \Delta t)\abs{\psi_T(x)}^2}
\end{equation}
where $G$ is the Green's function to the Fokker-Planck equation,
\begin{align}
    G&(x_{\text{new}}, x, \Delta x) \nonumber \\
    &= \frac{1}{(4\pi D \Delta t)^{3N/2}}\exp\left(-\frac{(x_{\text{new}} - x - D\Delta t F(x))^2}{4D\Delta t} \right).
\end{align}

\end{document}
