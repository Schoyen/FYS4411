\documentclass[
    a4paper, aps, twocolumn, floatfix, superscriptaddress,
    nofootinbib]{revtex4-1}

% It's nice to be able to write your own name
\usepackage[T1]{fontenc}
% Automatic clickable links
\usepackage{hyperref}
% SI-units
\usepackage{siunitx}
% Enhanced math formatting
\usepackage{amsmath}
% Extended math symbols
\usepackage{amssymb}
% Include proof environment
\usepackage{amsthm}
% Import physics package to include bra-ket
\usepackage{physics}
\usepackage{enumerate}
% Import the tensor package for tensors
\usepackage{tensor}
% Include dod and dpd fracs
\usepackage{commath}
% Include font for the identity operator
\usepackage{dsfont}
% Include Tikz
\usepackage{tikz}
% Tikz add-on
\usetikzlibrary{shapes,arrows}
\usetikzlibrary{positioning}
% Several figures in the same figure
\usepackage{subfig}
% Appendix
\usepackage[toc, page]{appendix}

% Macro for latin-letter vectors
\newcommand{\vf}{\mathbf}
% Macro for greek-letter vectors
\newcommand{\vfg}{\boldsymbol}

% Fast macro for real-numbers R
\newcommand{\R}{\mathbb{R}}
% Fast macro for complex-numbers C
\newcommand{\C}{\mathbb{C}}
% Fast macro for polynomial room P
\renewcommand{\P}{\mathbb{P}}
% New command for the identity operator
\newcommand{\1}{\mathds{1}}
% New command for the Lagrangian density
\newcommand{\cL}{\mathcal{L}}
% New command for the Hamiltonian density
\newcommand{\cH}{\mathcal{H}}
% Fast macro for partial differential tensors
\newcommand{\tpl}[1]{\tensor{\partial}{_#1}} % Lower
\newcommand{\tpu}[1]{\tensor{\partial}{^#1}} % Upper
% Fast macro for tensors
\newcommand{\te}[1]{\tensor{#1}}
% Fast macro for commutation and anti-commutation relations
\newcommand{\com}[2]{\left[#1, #2\right]}
\newcommand{\acom}[2]{\left\{#1, #2\right\}}

% Macros for writing auto-sized paranthesis, brackets and braces
\newcommand{\para}[1]{\left(#1\right)}
\newcommand{\brak}[1]{\left[#1\right]}
\newcommand{\brac}[1]{\left\{#1\right\}}

% Macro for creating orbital bra-ket's, i.e., bra-ket with paranthesis edges
\newcommand{\obra}[1]{( #1 \lvert}
\newcommand{\oket}[1]{\rvert #1 )}
\newcommand{\obraket}[2]{( #1 \lvert #2 )}

% Macro for expectation value
\newcommand{\expv}[1]{\langle #1 \rangle}

\newcommand{\half}{\frac{1}{2}}




\begin{document}

\title{Ground state energy of quantum dots using the coupled cluster method}
\author{Winther-Larsen, Sebastian Gregorius}
\homepage[Project code: ]{https://github.com/Schoyen/FYS4411}
\affiliation{University of Oslo}
\author{Schøyen, Øyvind Sigmundson}
\homepage[Project code: ]{https://github.com/Schoyen/FYS4411}
\affiliation{University of Oslo}
\date{\today}

\begin{abstract}
    Something about coupled-cluster... Preferably doubles.
\end{abstract}

\maketitle
\tableofcontents

\section{Introduction}
    In this project we will study the ground state energy of quantum dots.

\section{Theory}
    In this project we will study a system of $N$ interacting electrons. We will
    be looking at a Hamiltonian consisting of a one-body and a two-body part.
    The one-body part is given by
    \begin{align}
        h(\vf{r}_i)
        &= -\half\nabla_i^2 + \half\omega^2 \vf{r}_i^2,
    \end{align}
    where we use natural units $\hbar = c = e = 1$ and set the mass to unity.
    The two-body part is the Coulomb interaction potential.
    \begin{align}
        v(\vf{r}_i, \vf{r}_j)
        &= \frac{1}{\abs{\vf{r}_i - \vf{r}_j}}.
    \end{align}
    We thus get the total Hamiltonian
    \begin{align}
        H &= \sum_{i = 1}^N h(\vf{r}_i) + \sum_{i < j}^N v(\vf{r}_i, \vf{r}_j).
    \end{align}


\bibliography{references}

\end{document}
