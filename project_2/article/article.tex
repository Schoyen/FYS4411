\documentclass[
    a4paper, aps, twocolumn, floatfix, superscriptaddress,
    nofootinbib]{revtex4-1}

% It's nice to be able to write your own name
\usepackage[T1]{fontenc}
% Automatic clickable links
\usepackage{hyperref}
% SI-units
\usepackage{siunitx}
% Enhanced math formatting
\usepackage{amsmath}
% Extended math symbols
\usepackage{amssymb}
% Include proof environment
\usepackage{amsthm}
\usepackage{enumerate}
% Import the tensor package for tensors
\usepackage{tensor}
% Include dod and dpd fracs
\usepackage{commath}
% Include font for the identity operator
\usepackage{dsfont}
% Include Tikz
\usepackage{tikz}
% Tikz add-on
\usetikzlibrary{shapes,arrows}
\usetikzlibrary{positioning}
% Several figures in the same figure
\usepackage{subfig}
% Appendix
\usepackage[toc, page]{appendix}
\usepackage{simpler-wick}

% Macro for latin-letter vectors
\newcommand{\vf}{\mathbf}
% Macro for greek-letter vectors
\newcommand{\vfg}{\boldsymbol}

% Fast macro for real-numbers R
\newcommand{\R}{\mathbb{R}}
% Fast macro for complex-numbers C
\newcommand{\C}{\mathbb{C}}
% Fast macro for polynomial room P
\renewcommand{\P}{\mathbb{P}}
% New command for the identity operator
\newcommand{\1}{\mathds{1}}
% New command for the Lagrangian density
\newcommand{\cL}{\mathcal{L}}
% New command for the Hamiltonian density
\newcommand{\cH}{\mathcal{H}}
% Fast macro for partial differential tensors
\newcommand{\tpl}[1]{\tensor{\partial}{_#1}} % Lower
\newcommand{\tpu}[1]{\tensor{\partial}{^#1}} % Upper
% Fast macro for tensors
\newcommand{\te}[1]{\tensor{#1}}
% Fast macro for commutation and anti-commutation relations
\newcommand{\com}[2]{\left[#1, #2\right]}
\newcommand{\acom}[2]{\left\{#1, #2\right\}}

% Macros for writing auto-sized paranthesis, brackets and braces
\newcommand{\para}[1]{\left(#1\right)}
\newcommand{\brak}[1]{\left[#1\right]}
\newcommand{\brac}[1]{\left\{#1\right\}}

% Macro for creating orbital bra-ket's, i.e., bra-ket with paranthesis edges
\newcommand{\obra}[1]{( #1 \lvert}
\newcommand{\oket}[1]{\rvert #1 )}
\newcommand{\obraket}[2]{( #1 \lvert #2 )}

% Macro for expectation value
\newcommand{\expv}[1]{\langle #1 \rangle}

\newcommand{\half}{\frac{1}{2}}

\newcommand{\bra}[1]{\langle #1\lvert}
\newcommand{\ket}[1]{\rvert #1\rangle}
\newcommand{\braket}[2]{\langle #1 \vert #2 \rangle}

\newcommand{\acr}[1]{a_{#1}^{\dagger}}
\newcommand{\ade}[1]{a_{#1}}

\newcommand{\kslat}{\ket{\Phi_0}}
\newcommand{\bslat}{\bra{\Phi_0}}

\newcommand{\kcc}{\ket{\Psi_{\text{CC}}}}
\newcommand{\bcc}{\bra{\Psi_{\text{CC}}}}




\begin{document}

\title{Ground state energy of quantum dots using the coupled cluster method}
\author{Winther-Larsen, Sebastian Gregorius}
\homepage[Project code: ]{https://github.com/Schoyen/FYS4411}
\affiliation{University of Oslo}
\author{Schøyen, Øyvind Sigmundson}
\homepage[Project code: ]{https://github.com/Schoyen/FYS4411}
\affiliation{University of Oslo}
\date{\today}

\begin{abstract}
    Something about coupled-cluster... Preferably doubles.
\end{abstract}

\maketitle
\tableofcontents

\section{Introduction}
    %TODO: Add explanation on why the many-body problem becomes intractable.
    % This can be shown by demonstrating that the true wavefunction of the system
    % must be a linear combination of all possible Slater detemerminants built up
    % from permutations of the single particle functions in the basis. Thus the
    % energy equation becomes too large.
    In this project we will study the ground state energy of quantum dots.

\section{Theory}
    In this project we will study a system of $N$ interacting electrons. We will
    be looking at a Hamiltonian consisting of a one-body and a two-body part.
    The one-body part is given by
    \begin{align}
        h(\vf{r}_i)
        &= -\half\nabla_i^2 + \half\omega^2 \vf{r}_i^2,
    \end{align}
    where we use natural units $\hbar = c = e = 1$ and set the mass to unity.
    The two-body part is the Coulomb interaction potential.
    \begin{align}
        v(\vf{r}_i, \vf{r}_j)
        &= \frac{1}{\abs{\vf{r}_i - \vf{r}_j}}.
    \end{align}
    We thus get the total Hamiltonian
    \begin{align}
        H &= \sum_{i = 1}^N h(\vf{r}_i) + \sum_{i < j}^N v(\vf{r}_i, \vf{r}_j).
    \end{align}
    Working in a basis of $L$ single particle functions, $\{\ket{p}\}_{p =
    1}^L$. We define the reference Slater determinant as
    \begin{align}
        \ket{\Phi_0} &\equiv \ket{1, 2, \dots, N},
    \end{align}
    i.e., a tensorproduct of the $N$ first single particle functions, $\ket{i}$,
    of the system. We call these single particle functions \emph{occupied} as
    they are contained in the Slater determinant.  We will denote the occupied
    indices with $i, j, k, l, \dots \in \{1, \dots, N\}$, the \emph{virtual}
    states with $a, b, c, d, \dots \in \{N + 1, \dots, L\}$ and general indices
    with $p, q, r, s, \dots \in \{1, \dots, L\}$. In terms of sets of basis
    functions we can write this as
    \begin{align}
        \{\ket{p}\}_{p = 1}^L
        = \{\ket{i}\}_{i = 1}^N
        \cup
        \{\ket{a}\}_{a = N + 1}^L,
    \end{align}
    i.e., the general indexed states consists of both occupied and virtual
    states.

    \subsection{Second quantization}
        Employing the creation operators, $\acr{p}$, and the destruction
        operators, $\ade{p}$, we can write the Hamiltonian as
        \begin{align}
            H
            &=
            \sum_{pq}h_{q}^{p}\acr{p}\ade{q}
            + \sum_{pqrs}w_{rs}^{pq}\acr{p}\acr{q}\ade{s}\ade{r},
        \end{align}
        where the sums are general indices over all $L$ basis states and the
        matrix elements are defined as
        \begin{gather}
            h^{p}_{q} \equiv \bra{p}h\ket{q}, \\
            w_{rs}^{pq} \equiv \bra{pq}v\ket{rs}.
        \end{gather}
        Note that the two-body matrix elements are not antisymmetric yet.

    \subsection{The coupled cluster approximation}
        We approximate the true wavefunction, $\ket{\Psi}$, of the system by the
        coupled cluster wavefunction, $\ket{\Psi_{\text{CC}}}$, defined as
        \begin{align}
            \ket{\Psi_{\text{CC}}}
            &\equiv e^{T}\ket{\Phi_0}
            = \para{
                \sum_{i = 0}^n
                \frac{1}{n!}T^n
            }\ket{\Phi_0},
        \end{align}
        where the \emph{cluster operator}, $T$, is given by a sum of
        $p$-excitation operators
        \begin{align}
            T &= T_1 + T_2 + \dots + T_p \\
            &=
            \sum_{ia}t_i^a\acr{a}\ade{i}
            + \para{\frac{1}{2!}}^2\sum_{ijab}
            t_{ij}^{ab}\acr{a}\acr{b}\ade{i}\ade{j}
            + \dots,
        \end{align}
        where $t_{i\dots}^{a\dots}$
        We will be looking at the \emph{doubles} approximation, that is
        \begin{align}
            T = T_{\text{CCD}} \equiv T_2.
        \end{align}

    \subsection{Energy of the coupled cluster approximation}
        When we're going to compute the energy of a system using the coupled
        cluster approximation we would ideally want to find the expectation
        value of the energy using the coupled cluster wavefunction.
        \begin{align}
            E = \bcc H\kcc.
        \end{align}
        As it turns out, this is an uncomfortable way of finding the energy
        as $T \neq T^{\dagger}$. Instead we will define what we call the
        \emph{similarity transformed Hamiltonian}. We plug the coupled
        cluster wavefunction into the Schrödinger equation.
        \begin{align}
            H\kcc = E\kcc.
        \end{align}
        Next, we left multiply with the inverse of the cluster expansion,
        i.e.,
        \begin{align}
            e^{-T}H\kcc = e^{-T}E\kcc
            = E \kslat.
        \end{align}
        Projecting this equation on the reference state we get
        \begin{align}
            E = \bslat e^{-T}H\kcc
            = \bslat e^{-T}He^{T}\kslat,
        \end{align}
        where in the latter inner-product we have located the similarity
        transformed Hamiltonian defined by
        \begin{align}
            \bar{H} \equiv e^{-T}He^{T}.
            \label{eq:similarity_transformed_hamiltonian}
        \end{align}

        \subsubsection{Normal ordered Hamiltonian}
            To get to the coupled cluster equations we need to write the
            Hamiltonian on a normal ordered form. We use Wick's theorem on the
            Fermi vacuum\footnote{Fermi vacuum corresponds to the reference
            Slater.}. The one-body part of the Hamiltonian thus becomes
            \begin{align}
                h &= \sum_{pq}h^p_q\acr{p}\ade{q}
                = \sum_{pq}h^p_q\para{
                    \{\acr{p}\ade{q}\}
                    + \{
                        \wick{\c a_{p}^{\dagger} \c a_{q}}
                    \}
                }
                \\
                &= h_N + \sum_{pq}\varepsilon_i.
            \end{align}
            The two-body part yields
            \begin{align}
                v &= \sum_{pqrs}w^{pq}_{rs}\acr{p}\acr{q}\ade{s}\ade{r}
            \end{align}
            % TODO: Fill in the remainder. Preferably in the appendix.
            We can then write the Hamiltonian in terms of the reference energy
            and the normal ordered Hamiltonian.
            \begin{align}
                H = H_N + \bslat H\kslat.
            \end{align}
            The energy equation thus becomes
            \begin{align}
                E &= \bslat\bar{H}\kslat
                = E_{\text{ref}} + \bslat e^{-T}H_N e^T\kslat.
            \end{align}
            In this equation the unknowns are the cluster amplitudes


        By expanding the exponentials the similarity transformed Hamiltonian and
        recognizing the commutators we get the Baker-Campbell-Hausdorff
        expansion.
        \begin{align}
            \bar{H}
            &=
            H + \com{H}{T} + \half\com{\com{H}{T}}{T} + \dots.
        \end{align}

\bibliography{references}

\end{document}
